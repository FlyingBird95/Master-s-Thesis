%********************************************************************
% Appendix
%*******************************************************
\chapter{Demo application} 
\label{ch:demo_application}
The proposed demo application described in \autoref{sec:demo_app} uses three different nodes and a root.

\section{Root}
In order to start the root, the following command is used. Note that the dashboard can be viewed at \texttt{http://localhost:14100}, or using a public IP address.

\begin{lstlisting}[language=bash, caption={Root command}]
#!/bin/bash
docker run --rm -d \
  --name root \
  -p 14100:14100 \
  -e TYPE=ROOT \
  dadvisor/container
\end{lstlisting}

\section{Supernode} \label{sec:deploy_supernode}
In order to start one supernode with two web-services, use the following command.

\begin{lstlisting}[language=bash, caption={Supernode command}]
#!/bin/bash
docker run --rm -d \
  --name=dadvisor \
  --net=host \
  --volume=/:/rootfs:ro \
  --volume=/var/run:/var/run:ro \
  --volume=/sys:/sys:ro \
  --volume=/var/lib/docker/:/var/lib/docker:ro \
  --volume=/dev/disk/:/dev/disk:ro \
  --env TRACKER=http://root:14100 \ # configurable
  --env TYPE=SUPERNODE \
  dadvisor/container

docker run --rm -d \
  --name web1 \
  -p 1000:6000 \
  -e ITERATIONS=10000000 \
  dadvisor/web

docker run --rm -d \
  --name web3 \
  -p 2000:6000 \
  -e ITERATIONS=10000000 \
  dadvisor/web
\end{lstlisting}

\section{Node 1}
The following commands can be used for deploying a node with two request-services. Node that one request communicates with the supernode and one request communicates with node 2.
\begin{lstlisting}[language=bash, caption={Node 1 command}]
#!/bin/bash
docker run --rm -d \
--name=dadvisor \
--net=host \
--volume=/:/rootfs:ro \
--volume=/var/run:/var/run:ro \
--volume=/sys:/sys:ro \
--volume=/var/lib/docker/:/var/lib/docker:ro \
--volume=/dev/disk/:/dev/disk:ro \
--env TRACKER=http://root:14100 \ # configurable
--env TYPE=NODE \
dadvisor/container

docker run --rm -d \
  --name req1 \
  -e HOST=http://supernode:1000 \ # configurable
  dadvisor/req

docker run --rm -d \
  --name req2 \
  -e HOST=http://node2:3000 \
  dadvisor/req
\end{lstlisting}

\section{Node 2}
The following commands can be used for deploying a node with one request-service and one web service. Additional, three Cassandra containers are started, as they are known to use a lot of memory.
\begin{lstlisting}[language=bash, caption={Node 2 command}]
#!/bin/bash
docker run --rm -d \
  --name=dadvisor \
  --net=host \
  --volume=/:/rootfs:ro \
  --volume=/var/run:/var/run:ro \
  --volume=/sys:/sys:ro \
  --volume=/var/lib/docker/:/var/lib/docker:ro \
  --volume=/dev/disk/:/dev/disk:ro \
  --env TRACKER=http://root:14100 \ # configurable
  --env TYPE=NODE \
  dadvisor/container

docker run --rm -d \
  --name req3 \
  -e HOST=http://supernode:2000 \ # configurable
  dadvisor/req

docker run --rm -d \
  --name web2 \
  -p 3000:6000 \
  -e ITERATIONS=10000000 \
  dadvisor/web

curl -sS https://gist.githubusercontent.com/naumanbadar/aad6a25974b30adcb3c89b5f868627da/raw/b72129a8b3d5eff7597c4daf01267060dc1c30c7/docker-compose-cassandra-cluster.yml > docker-compose.yml

docker-compose up -d
\end{lstlisting}