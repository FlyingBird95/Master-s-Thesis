%************************************************
\chapter{Conclusion and Future work}\label{ch:conclusion}
%************************************************
This chapter concludes the thesis by providing a small summary of the work performed. Furthermore, it ends with a section about the future work (\autoref{sec:future_work}).

\section{Conclusion}
This research attempts to provide insights into the costs of Cloud Computing. In order to do so, the costs are divided into two categories, which are \textit{effective cost} and \textit{waste cost}. In order to monitor theses costs, a solution is proposed in \autoref{ch:design}. However, the Cloud is composed of five essential characteristics, which are important requirements for the proposed solution.\\

\noindent
\autoref{ch:related_work} describes the related literature in the field of Cloud Computing Monitoring. It lists the open issues, which includes the lack of custom algorithms and techniques that provide effective summaries. The related literature is grouped into three categories: resource selection, architecture, and estimating cost. For all of these categories, the relevant papers are discussed. This chapter ends with explaining the popular open-source stack for Service Monitoring, which consists of cAdvisor, Prometheus and Grafana.\\

\noindent
\autoref{ch:design} proposes a solution, based on the architectural requirements from \autoref{ch:introduction}. Due to the limitation of Prometheus, the architecture is constructed in a hierarchical model. This model consists of three levels, for which the root collects all the available data. The supernodes, and nodes are used as a probe, which collects all the data about the monitored instances. Furthermore, a pricing model is proposed, which divides the costs into the two categories, by distributing the \textit{waste cost} on a containerized level. In order to compute the waste, three approaches are proposed, of which the \textit{linear inverse distribution} is considered the most optimal.\\

\noindent
Implementation details are described in \autoref{ch:implementation}, such as the underlying communication, and the internal commands that are used to collect and resolve network packets. Furthermore, several configuration details are discussed that can be used to tweak the solution, to be functional across multiple Cloud-based Applications. The system is presented using a demo application. The demo application shows the strong points of the system, including the interface.\\

\noindent
An important aspect of this thesis is to evaluate the proposed solution in an empirical manner. This is performed by deploying the solution at an industrial company. This company meets the requirements for evaluating the developed solution, because it is a relatively large system with load. \autoref{ch:case_study} presents the limitations that showed up during the performed case study. However, this also revealed the relation between the waste cost of a container group, and the number of containers attached to that group. The deployment process of the system is evaluated by an interview with a technical employee.\\

\noindent
Apart from the interview, the solution is analysed by evaluating the requirements of the system. Furthermore, the entire pricing model is evaluated and lead to the conclusion that there are several improvements possible. The evaluation chapter ends with answering the sub questions. These questions can be used to answer the research question of this thesis:

\begin{quote}
    \begin{itemize}
        \item[\textbf{Q1}: ]\textit{How can the cost and waste monitoring of a Cloud-based Application be offered in a scalable manner?}
    \end{itemize}
\end{quote}

\noindent
The proposed solution starts with monitoring the resources of the individual components using a distributed architecture. By proposing a pricing model can the resources be converted into the \textit{effective cost}. This can be computed on a containerized level. Using an algorithm, the unused resources are divided over the available containers with a \textit{linear inverse distribution}. This can be used to compute the \textit{waste cost}. The results can be aggregated by assigning multiple containers into a cost-group. The information is presented at the root node of the system, using a Grafana dashboard. This leads to a scalable manner of monitoring a Cloud-based application.

\section{Future work} \label{sec:future_work}
As explained in the case study, the proposed solution has several limitations that require further work. They are listed below:

\begin{itemize}
    \item The industrial company uses Docker Weave for the intra-host communication. Therefore, they communicate using the weave network, which assigns new IP-addresses to the containers. The proposed solution was unable to resolve the network communication send over this network. Therefore, future work is needed to provide a mapping from IP address to container ID. This mapping is necessary for resolving the network data, so that the external amount can be calculated.
    \item The proposed solution is able to monitor a Cloud-based Application that is executed by Docker or Docker-Compose. The scope of this research is therefore limited to these technologies. As a consequence, the system must be evaluated and adapted such that it works with other technologies, such as Docker Swarm\footnote{See \url{https://docs.docker.com/engine/swarm/}} or Kubernetes\footnote{See \url{https://kubernetes.io/}}.
    \item The industrial company has not provided any financial details of running their services into the Cloud. Therefore, the evaluation whether the \textit{effective cost} and \textit{waste cost} are inline with the actual cost is insufficient. This implies that future work is needed to verify the proposed pricing model.
\end{itemize}