%************************************************
\chapter{Introduction}\label{ch:introduction}
%************************************************
The Cloud is a set of different types of hardware and software that work collectively to deliver many aspects of computing to the end-user as an online service \cite{lenovo_cloud}. Cloud Computing is the use of hardware and software to deliver a service over a network, usually the Internet. The NIST definition of cloud computing is the following: ``Cloud Computing is a model for enabling ubiquitous, convenient, on-demand network access to a shared pool of configurable computing resources (e.g., networks, servers, storage, applications, and services) that can be rapidly provisioned and released with minimal management effort or service provider interaction.''\cite{nist:cloud_computing} This cloud model is composed of five essential characteristics and three service models. These are:
\begin{enumerate}
    \item On-demand self-service
    \item Broad network access
    \item Resource pooling
    \item Rapid elasticity
    \item Measured service
\end{enumerate}

\noindent
And the three service models are:
\begin{enumerate}
    \item Software as a Service (SaaS): The capability provided to the consumer is to use the provider's applications running on a cloud infrastructure.
    \item Platform as a Service (Paas): The capability provided to the consumer is to deploy onto the cloud infrastructure consumer-created or acquired applications created using programming languages, libraries, services, and tools supported by the provider.
    \item Infrastructure as a Service (IaaS): The capability provided to the consumer is to provision processing, storage, networks, and other fundamental computing resources where the consumer is able to deploy and run arbitrary software, which can include operating systems and applications.
\end{enumerate}

\noindent
When developing a monitoring application that targets to deploy in the cloud, it is important to keep the essential characteristics in mind. The third characteristic is resource pooling, and this is especially interesting as Cloud Systems automatically control and optimize resource use by leveraging a metering capability at some level of abstraction appropriate to the type of service.

\section{Cloud Computing Monitoring} \label{sec:intro_monitoring}
Monitoring of Cloud Computing Systems exists from the beginning of Cloud Computing. It is a key tool for controlling and managing hardware and software infrastructures \cite{aceto2013cloud}. Additionally, it provides information and Key Performance Indicators (KPIs) for both platforms and applications.\\

\noindent
The providers of Cloud Computing services charge their customers for what they use. This is determined by monitoring metrics and their monitoring frequency. Monitoring cost is usually proportional to the number of monitored metrics and monitoring frequency, which takes up $18\%$ of total running cost for cloud computing systems \cite{wang2018self}. On one hand, administrators and customers want to decrease the number of monitored metrics and decrease the monitoring frequency to reduce their costs; on the other hand, less monitored metrics and lower monitoring frequency result in less monitoring data, making it hard to locate faults efficiently. It is therefore the challenge to find a balance between the monitoring frequency and selecting the appropriate metrics to reduce the monitoring costs, while gaining important insight into the performance.\\

\noindent
A key feature of Cloud Computing is to pay proportionally to the use of the service with different metrics and different granularity, according to the type of service and the price model adopted. For example, in SaaS the number of users of a system can be a reference of billing criteria. In PaaS, the CPU utilization, or the task completion time is interesting. And lastly, for IaaS, the number of VM (and the amount of resources) can be used as a billing criteria \cite{katsaros2011building}. Monitoring is important for the Consumer side for verifying his own usage and to compare different Providers. This results in choosing the best Provider, according to the metrics of the Consumer.\\

\noindent
Another key feature of Cloud Computing is scalability. A monitoring system is scalable if it can cope with a large number of probes \cite{clayman2010monitoring}. Such property is very important in Cloud Computing scenarios due to the large number of parameters to be monitored about a huge number of resources \cite{aceto2013cloud}. A monitoring system is resilient when the persistence of service delivery can justifiably be trusted when facing changes \cite{laprie2008dependability}, that basically means to withstand a number of component failures while continuing to operate normally. Both properties are critical for a Cloud Computing Monitoring System as it leads to incorrect results otherwise.\\

\noindent
A lifecycle activity which relies on monitoring with respect to this thesis is the performance management. Despite the attention paid by Providers, some Cloud nodes may attain performance orders of magnitude worse than other nodes \cite{fox2009above}. From a Consumer's perspective, monitoring the perceived performance is necessary to adapt to the changes or to apply corrective measures \cite{aceto2013cloud}. In the remainder of this thesis, performance monitoring is considered as important, and will be discussed further in \autoref{ch:related_work}. The next section explains the importance of monitoring the actual costs of deploying and running Cloud Computing Systems.

\section{Estimating \textit{actual} costs} \label{sec:intro_pricing}
As discussed above, there is a strong need for monitoring Cloud Computing at both the providers side, as well as the consumer side. This monitoring can be analysed with respect to both the performance, as well as the pricing. The concept is fairly simple. If the Cloud Consumer purchases more resources, then a better performance can be expected. However, most Consumers have a maximum budget to spend, while demanding a certain Service-Level Agreement. It is up to the Consumer to find a certain balance between the cost and performance of his deployed system. This reveals that it is interesting to compare the actual costs of running a certain deployment, and to compare this to the gained performance. When describing the cost, there is a difference between the actual costs that are paid for the Cloud-based application, and the effective costs, which is an amount always smaller or equal to this actual cost. The effective cost can be described as a measure of a certain amount of resources that are actually used, expressed in a currency. What is left, is a certain amount of currency that is not used at all, and thus are wasted. It is up to the cloud consumer to reduce this amount to a minimum, while meeting the system demands at all time. Mathematically, the total cost can be formulated by the sum of the effective cost and the waste:

\begin{equation}\label{eq:intro}
  c_\text{actual} = c_\text{effective} + c_\text{waste}  
\end{equation}

\noindent
Expressing the actual cost in terms of effective cost and waste is desired, as it easily summarizes the main target of a Cloud Computing Deployment. These two values, and the waste-ratio (which is $c_\text{waste} / c_\text{actual}$), are an effective manner of expressing the amount of resources that can be economized. However, when computing these costs, there is a certain price involved. Monitoring the actual costs of a Cloud deployment is rather complicated, as it depends on how the pricing is modelled and around what economic principles it is based. In \cite{samimi2011review} the basic core principles are presented and they provide a comparative review of the latest and most appropriate economic and pricing models applicable to grid and Cloud Computing. Therefore, it is not the focus of this thesis to reveal the actual costs of running a certain deployment, but rather use several metrics to estimate this. Eventually, this pricing model will be used to compute the total cost, and the effective cost. With a simple rewriting of \autoref{eq:intro}, the waste can be computed. Note that all these variables can be computed per VM. For a given deployment that consists of multiple nodes, the total actual cost is the sum of all nodes. This also holds for the total waste and total effective cost.

\section{Research question} \label{sec:research_question}
The purpose of this research is based on need for monitoring Cloud Computing Systems in terms of computing actual costs and waste of these deployments. Therefore, the research question can be stated as:

\begin{quote}
    \begin{itemize}
        \item[\textbf{Q1}: ]\textit{How can the cost and waste monitoring of a Cloud-based Application be offered in a scalable manner?}
    \end{itemize}
\end{quote}

\noindent
This research question can be answered by splitting this down into several sub questions:

\begin{quote}
    \begin{itemize}
        \item[\textbf{Q2}: ]\textit{How can the utilization of a Cloud-based application be monitored in an effective manner?}
        \item[\textbf{Q3}: ]\textit{What is an accurate pricing model for estimating the cost of a Cloud-based application?}
        \item[\textbf{Q4}: ]\textit{What is an effective approach for computing the waste of a Cloud-based application?}
        \item[\textbf{Q5}: ]\textit{How can the cost and waste be presented to the user in a scalable and effective manner?}
    \end{itemize}
\end{quote}

\section{Audience and overview} \label{sec:audience}
Each of the sub questions from \autoref{sec:research_question} are evaluated in the remainder of this thesis. The questions are relevant for the Cloud Computing Consumer, as they lack appropriate methods for estimating the actual costs and waste of their deployment. This research can therefore be used by the consumer to optimize the waste, by more efficiently using the available resources. \\

\noindent
The remainder of this thesis is divided into the following chapters. \autoref{ch:related_work} states the related work, as well as existing technologies that can be used for providing a solution. \autoref{ch:design} explains the design and requirements of the desired solution. \autoref{ch:implementation} provides information about the methods that are used for the proposed solution. It also proposes a demo application, that is used for evaluating the system. This solution is deployed in a case study, which is described in \autoref{ch:case_study}.
The results of the demo application are evaluated in \autoref{ch:evaluation}. This thesis ends with \autoref{ch:conclusion} that provides a conclusion and future work.